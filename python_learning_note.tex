% Created 2019-09-21 Sat 01:07
% Intended LaTeX compiler: pdflatex
\documentclass[11pt]{article}
\usepackage[utf8]{inputenc}
\usepackage[T1]{fontenc}
\usepackage{graphicx}
\usepackage{grffile}
\usepackage{longtable}
\usepackage{wrapfig}
\usepackage{rotating}
\usepackage[normalem]{ulem}
\usepackage{amsmath}
\usepackage{textcomp}
\usepackage{amssymb}
\usepackage{capt-of}
\usepackage{hyperref}
\author{Li Gao-Yang}
\date{\today}
\title{python learning note}
\hypersetup{
 pdfauthor={Li Gao-Yang},
 pdftitle={python learning note},
 pdfkeywords={},
 pdfsubject={},
 pdfcreator={Emacs 26.3 (Org mode 9.1.9)},
 pdflang={English}}
\begin{document}

\maketitle
\tableofcontents

\section{python support in org mode}
\label{sec:orgf1ebe3f}
\subsection{python code block}
\label{sec:org0e220ff}
\begin{verbatim}
return 2/3
\end{verbatim}

\begin{verbatim}
print(2/3)
\end{verbatim}

\begin{verbatim}
print(2/3)
\end{verbatim}

\subsection{inline calling}
\label{sec:org706d256}
Two plus two equals \texttt{4}
\subsection{plotting}
\label{sec:org9421ce3}
\begin{verbatim}
import matplotlib, numpy
#matplotlib.use('Agg')
import matplotlib.pyplot as plt
fig=plt.figure(figsize=(4,2))
x=numpy.linspace(-15,15)
plt.plot(numpy.sin(x)/x)
fig.tight_layout()
plt.savefig('python-matplot-fig.png')
return 'python-matplot-fig.png' # return filename to org-mode
\end{verbatim}

\section{A primer on scientific programming with python}
\label{sec:orgd2fc477}
\subsection{using a program as a calculator}
\label{sec:org724ffad}
\begin{verbatim}
print(5*0.6 - 0.5*9.81*0.6**2)
\end{verbatim}
\subsection{using variables}
\label{sec:orgd8edf9d}
\begin{verbatim}
v0 = 5
g = 9.81
t = 0.6
y = v0*t - 0.5*g*t**2
print(y)
\end{verbatim}
\subsection{{\bfseries\sffamily Important} print syntax}
\label{sec:orgbd7980c}
\begin{itemize}
\item State "Important"  from ""           \textit{[2018-05-06 Sun 13:48]}
\end{itemize}
\begin{verbatim}
v0 = 5
g = 9.81
t = 0.6
y = v0*t - 0.5*g*t**2
print('At t=%g s, the height of the ball is %.2f m.' % (t,y))
\end{verbatim}
\begin{verbatim}
print("""y(t) is
the position of
our ball.""")
print()
print('y(t) is\nthe position of\nour ball')
\end{verbatim}
\section{Liaoxuefeng python3}
\label{sec:orgd73449a}
\subsection{python 基础}
\label{sec:org66aaee2}
以\#开头的语句是注释
每一行都是一个语句
当语句以冒号:结尾时,缩进的语句视为代码块。
应该始终坚持使用4个空格的缩进。
Python程序是大小写敏感的

\begin{verbatim}
a = 100
if a >= 0:
    print(a)
else:
    print(-a)
\end{verbatim}
\subsubsection{数据类型}
\label{sec:orgdb0935a}
\begin{itemize}
\item 整数
Python可以处理任意大小的整数
\item 浮点数
\item 字符串
\begin{itemize}
\item 字符串是以单引号'或双引号"括起来的任意文本
\item 如果'本身也是一个字符,那就可以用""括起来
\item 如果字符串内部既包含'又包含",可以用转义字符$\backslash$来标识
\item 允许用r''表示''内部的字符串默认不转义
\item 允许用'''\ldots{}'''的格式表示多行内容
\end{itemize}
\item 布尔值
\begin{itemize}
\item 只有True、False两种值
\item and or not
\end{itemize}
\item 空值none
\item 变量,动态语言与静态语言
\item 常量
\begin{itemize}
\item 通常用全部大写的变量名表示常量
\end{itemize}
\end{itemize}
\begin{verbatim}
print('I\'m ok.')
print('I\'m learning\nPython.')
print('\\\n\\')
print(r'''hello,\n
world''')
\end{verbatim}
\subsubsection{{\bfseries\sffamily Important} 字符串与编码}
\label{sec:orgeb048d4}
\begin{itemize}
\item State "Important"  from ""           \textit{[2018-05-03 Thu 10:49]}
\end{itemize}

\begin{itemize}
\item \%运算符就是用来格式化字符串的。
\item 在字符串内部,\%s表示用字符串替换,
\item \%d表示用整数替换,
\item 有几个\%?占位符,后面就跟几个变量或者值,顺序要对应好。
\item 如果只有一个\%?,括号可以省略。
\item 格式化整数和浮点数还可以指定是否补0和整数与小数的位数
\item \%s永远起作用,它会把任何数据类型转换为字符串

\item \%d	整数
\item \%f	浮点数
\item \%s	字符串
\item \%x	十六进制整数
\end{itemize}

\begin{verbatim}
print('Hello, %s' % 'world')
print('Hi, %s, you have $%d.' % ('Michael', 1000000))
print('%4d-%0004d' % (3, 1))
print('%.2f' % 3.1415926)
\end{verbatim}
\subsubsection{list与tuple \textit{<2018-05-03 Thu>}}
\label{sec:orgd7cdd9f}
\begin{enumerate}
\item list
\label{sec:orgba0fc88}
\begin{itemize}
\item 索引是从0开始
\item 用-1做索引,直接获取最后一个元素
\item 往list中追加元素到末尾, classmates.append('Adam')
\item 把元素插入到指定的位置, classmates.insert(1,'Jack')
\item 删除list末尾的元素,用pop()
\item 删除指定位置的元素,用pop(i)
\end{itemize}
\begin{verbatim}
classmates = ['Michael', 'Bob', 'Tracy']
print(classmates)
print(len(classmates))
print(classmates[-3])
classmates.append('Adam')
print(classmates)
\end{verbatim}
\item tuple
\label{sec:orgac0af53}
tuple和list非常类似,但是tuple一旦初始化就不能修改
\begin{verbatim}
classmates = ('Michael', 'Bob', 'Tracy')
t = (1,)
\end{verbatim}
\end{enumerate}
\subsubsection{{\bfseries\sffamily Important} 条件判断 \textit{<2018-05-04 Fri>}}
\label{sec:org816d645}
\begin{itemize}
\item State "Important"  from ""           \textit{[2018-05-04 Fri 09:30]}
\end{itemize}
\begin{itemize}
\item if语句的完整形式
if <条件判断1>:
    <执行1>
elif <条件判断2>:
    <执行2>
elif <条件判断3>:
    <执行3>
else:
    <执行4>
\end{itemize}
它是从上往下判断,如果在某个判断上是True,
把该判断对应的语句执行后,就忽略掉剩下的elif和else.
\#\#\#\#\#\#\#\#\#\#\#\#\#\#\#\#\#\#\#\#\#\#\#\#\#\#\#\#\#\#\#\#\#\#\#\#\#\#\#\#\#\#\#\#\#\#\#\#\#\#\#\#\#\#\#\#\#\#\#\#\#\#\#\#\#\#\#\#\#\#\#\#\#\#
\begin{itemize}
\item 只要条件为非零数值、非空字符串、非空list等,就判断为True,否则为False。\#
\end{itemize}
\#\#\#\#\#\#\#\#\#\#\#\#\#\#\#\#\#\#\#\#\#\#\#\#\#\#\#\#\#\#\#\#\#\#\#\#\#\#\#\#\#\#\#\#\#\#\#\#\#\#\#\#\#\#\#\#\#\#\#\#\#\#\#\#\#\#\#\#\#\#\#\#\#\#
\subsubsection{循环 \textit{<2018-05-04 Fri>}}
\label{sec:org860ae57}
\begin{itemize}
\item for x in list:
\item while x :
\item break
\item continue
\end{itemize}
\begin{verbatim}
print(range(5))
print(list(range(5)))
for i in range(5):
    print(i)
\end{verbatim}
\subsubsection{使用dict \textit{<2018-05-05 Sat>}}
\label{sec:org42bc701}
dict全称dictionary,在其他语言中也称为map,
使用键-值(key-value)存储,具有极快的查找速度。
把数据放入dict的方法,除了初始化时指定外,还可以通过key放入
要避免key不存在的错误,有两种办法:
\begin{itemize}
\item 通过in判断key是否存在
\item 通过dict提供的get()方法,如果key不存在,
可以返回None,或者自己指定的value
\end{itemize}
要删除一个key,用pop(key)方法,对应的value也会从dict中删除
请务必注意,dict内部存放的顺序和key放入的顺序是没有关系的
\#\#\#\#\#\#\#\#\#\#\#\#\#\#\#\#\#\#\#\#\#\#\#\#\#\#\#\#\#\#\#\#\#\#\#\#\#\#\#\#\#\#\#\#\#\#\#\#\#\#\#\#\#\#\#\#\#\#
\#\#  和list比较,dict有以下几个特点:
\#\#  查找和插入的速度极快,不会随着key的增加而变慢;
\#\#  需要占用大量的内存,内存浪费多。
\#\#
\#\#  而list相反:
\#\#  查找和插入的时间随着元素的增加而增加;
\#\#  占用空间小,浪费内存很少。
\#\#  所以,dict是用空间来换取时间的一种方法。
\#\#\#\#\#\#\#\#\#\#\#\#\#\#\#\#\#\#\#\#\#\#\#\#\#\#\#\#\#\#\#\#\#\#\#\#\#\#\#\#\#\#\#\#\#\#\#\#\#\#\#\#\#\#\#\#\#\#
\begin{itemize}
\item 哈希算法(Hash)
\item 作为key的对象就不能变。在Python中,字符串、整数等都是不可变的,
因此,可以放心地作为key。而list是可变的,就不能作为key
\end{itemize}
\begin{verbatim}
d={'Michael': 95, 'Bob':75, 'Tracy':85}
d['Adam'] = 67
print(d['Michael'])
print('Adam' in d)
print(d.get('Tom', 0))
print(d.pop('Bob'))
print(d)
\end{verbatim}
\subsubsection{使用set \textit{<2018-05-05 Sat>}}
\label{sec:org8f13606}
\begin{itemize}
\item set和dict类似,也是一组key的集合,但不存储value。
由于key不能重复,所以,在set中,没有重复的key。
\item 通过add(key)方法可以添加元素到set中
\item 过remove(key)方法可以删除元素
\end{itemize}
\begin{verbatim}
s = set([1, 2, 3])
print(s)
s1 = set([1, 2, 3])
s2 = set([2, 3, 4])
print(s1 & s2)
print(s1 | s2)
print(set((1,2,3)))
# print(set((1,[2,3])))
new_dict={'Wang':59}
new_dict['Ligy']=100
print(new_dict)
\end{verbatim}

\subsection{函数}
\label{sec:org55b5665}
\subsubsection{调用函数 \textit{<2018-05-06 Sun>}}
\label{sec:orgc0814b7}
\begin{itemize}
\item 可以在交互式命令行通过help(abs)查看abs函数的帮助信息
\end{itemize}
\subsubsection{数据类型转换 \textit{<2018-05-06 Sun>}}
\label{sec:org923b99d}
\begin{itemize}
\item int('123')
\item float('12.4')
\item str(1.2)
\item bool(1)
\item bool('')
\item 函数名其实就是指向一个函数对象的引用,
完全可以把函数名赋给一个变量,相当于给这个函数起了一个“别名
\end{itemize}
\begin{verbatim}
print(bool("2"))
a = abs
print(a(-9))
tuple_={'255':hex(int('255'))}
tuple_['1000']=hex(int('1000'))
print(tuple_['255'],tuple_['1000'])
print('255的16进制表示为%s\n1000的16进制表示为%s' % (hex(255), hex(1000)))
\end{verbatim}
\subsubsection{定义函数 \textit{<2018-05-06 Sun>}}
\label{sec:org6a59ec4}
\begin{itemize}
\item 定义一个函数要使用def语句,依次写出函数名、括号、
括号中的参数和冒号:,然后,在缩进块中编写函数体,
函数的返回值用return语句返回。
\item 函数体内部的语句在执行时,一旦执行到return时,
函数就执行完毕,并将结果返回
\item 如果没有return语句,函数执行完毕后也会返回结果,只是结果为None
\item 果你已经把my\(_{\text{abs}}\)()的函数定义保存为abstest.py文件了,
那么,可以在该文件的当前目录下启动Python解释器,
用from abstest import my\(_{\text{abs来导入my}}\)\(_{\text{abs}}\)()函数,
注意abstest是文件名(不含.py扩展名)
\item pass语句什么都不做,那有什么用?实际上pass可以用来作为占位符,
比如现在还没想好怎么写函数的代码,就可以先放一个pass,
让代码能运行起来。
\end{itemize}
\begin{verbatim}
def my_print(x):
    print('what\'s up man?')
    # return x*x
a=my_print(3)
print(a)
\end{verbatim}
\begin{enumerate}
\item 空函数
\label{sec:org1ba106f}
\item 参数检查
\label{sec:orgaa9edff}
数据类型检查可以用内置函数isinstance()实现。
\begin{verbatim}
def my_abs(x):
    if not isinstance(x, (int, float)):
        raise TypeError('bad operand type')
    if x >= 0:
        return x
    else:
        return -x
my_abs('A')
\end{verbatim}
\item 返回多个值
\label{sec:org87da2d0}
Python的函数返回多值其实就是返回一个tuple,但写起来更方便
return nx, ny
\end{enumerate}
\subsubsection{函数的参数 \textit{<2018-05-07 Mon>}}
\label{sec:org65056ba}
\begin{itemize}
\item 位置参数 power(x), power(x, n), 调用函数时,
传入的两个值按照位置顺序依次赋给参数x和n
\item 默认参数 power(x, n=2)
设置默认参数时,有几点要注意:
\begin{itemize}
\item 一是必选参数在前,默认参数在后,否则Python的解释器会报错
\item 二是当函数有多个参数时,把变化大的参数放前面,变化小的参数放后面。
变化小的参数就可以作为默认参数。
\item 有多个默认参数时,调用的时候,既可以按顺序提供默认参数,
比如调用enroll('Bob', 'M', 7)用在参数age上,
city参数由于没有提供,仍然使用默认值。
\item 也可以不按顺序提供部分默认参数。
当不按顺序提供部分默认参数时,需要把参数名写上。
\item 定义默认参数要牢记一点:默认参数必须指向不变对象!
\end{itemize}
\end{itemize}
\begin{verbatim}
def add_end(L=[]):
    if L is None:
        L = []
    L.append('END')
    return L
print(add_end())
print(add_end())

def add_end(L=None):
    if L is None:
        L = []
    L.append('END')
    return L
print(add_end())
print(add_end())
\end{verbatim}
\begin{enumerate}
\item {\bfseries\sffamily TODO} 可变参数
\label{sec:orga7c69ed}
\begin{itemize}
\item State "TODO"       from ""           \textit{[2018-05-07 Mon 19:18]}
\end{itemize}
\item {\bfseries\sffamily TODO} 关键字参数
\label{sec:orgd8c2c9b}
\begin{itemize}
\item State "TODO"       from ""           \textit{[2018-05-07 Mon 19:19]}
\end{itemize}
\item {\bfseries\sffamily TODO} 命名关键字参数
\label{sec:orgcf8c1b0}
\begin{itemize}
\item State "TODO"       from ""           \textit{[2018-05-07 Mon 19:20]}
\end{itemize}
\item {\bfseries\sffamily TODO} 参数组合
\label{sec:org015fc41}
\begin{itemize}
\item State "TODO"       from ""           \textit{[2018-05-07 Mon 19:20]}
\end{itemize}
\item {\bfseries\sffamily TODO} 小结
\label{sec:org64fc24a}
\begin{itemize}
\item State "TODO"       from ""           \textit{[2018-05-07 Mon 19:20]}
\end{itemize}
\end{enumerate}
\subsubsection{{\bfseries\sffamily TODO} 递归函数 \textit{<2018-05-07 Mon>}}
\label{sec:org0428bb8}
\begin{itemize}
\item State "TODO"       from ""           \textit{[2018-05-07 Mon 19:24]}
\end{itemize}
\subsection{高级特性}
\label{sec:orge3361b9}
在Python中,代码不是越多越好,而是越少越好。
代码不是越复杂越好,而是越简单越好。
\subsubsection{slice 切片 \textit{<2018-05-07 Mon>}}
\label{sec:orgd070269}
L = ['Michael', 'Sarah', 'Tracy', 'Bob', 'Jack']

\begin{itemize}
\item 取前3个元素 L[0:3]
\item 倒数切片 L[-2:]
\item L[:2] 索引 L[0], L[1]
\item 空的list也可以用L[:1]的方式索引
\item 倒序索引 L[-1::-1]
\end{itemize}
\begin{verbatim}
L = ['Michael', 'Sarah', 'Tracy', 'Bob', 'Jack']
print(len(L))
#print(L[:3])
#print(L[1:3])
#print(L[-5])
print(L[-4:100])
print(L[-1::-1])
print(L[0])
N = list(range(40))
print(N[:20:3])
print(N[:])
print((0, 1, 2, 3, 4, 5, 6, 7, 8)[:3])
print('abcdefghi'[::2])
L1=''
print(L1[:-1])
print(len(L1))
def trim(str_):
    if not isinstance(str_, str):
        raise TypeError('bad operand type')

    while str_[:1]==' ':
        str_=str_[1:]
        print(str_)
    while str_[-1:]==' ':
        str_=str_[-len(str_):-2]
        print(str_)
    return str_
\end{verbatim}
\subsubsection{iteration 迭代 \textit{<2018-05-08 Tue>}}
\label{sec:org9273557}
\begin{itemize}
\item 如果给定一个list或tuple,我们可以通过for循环来遍历这个list或tuple,
这种遍历我们称为迭代(Iteration)。
\item 在Python中,迭代是通过for \ldots{} in来完成的,而很多语言比如C语言,
迭代list是通过下标完成的,比如Java代码
\item list这种数据类型虽然有下标,但很多其他数据类型是没有下标的,但是,
只要是可迭代对象,无论有无下标,都可以迭代,比如dict就可以迭代
\item 默认情况下,dict迭代的是key。如果要迭代value,可以用for value in d.values(),
如果要同时迭代key和value,可以用for k, v in d.items()。
\item 由于字符串也是可迭代对象,因此,也可以作用于for循环
\item 如何判断一个对象是可迭代对象呢?
方法是通过collections模块的Iterable类型判断
\item enumerate函数可以把一个list变成索引-元素对,
这样就可以在for循环中同时迭代索引和元素本身
\item 上面的for循环里,同时引用了两个变量,在Python里是很常见的
,比如下面的代码
\end{itemize}
\begin{verbatim}
d = {'a': 1, 'b': 2, 'c': 3}
for k, v in d.items():
    print(k, v)
for ch in 'AbC':
    print(ch, 'Abc'[::-1])
from collections import Iterable
print(isinstance('abc', Iterable)) # str是否可迭代
for i, value in enumerate(['A', 'B', 'C']):
    print(i, value)
for x, y in [(1, 'a'), (2, d), (3, v)]:
    print(x, y)
\end{verbatim}

\subsubsection{list comprehensions 列表生成式 \textit{<2018-05-21 Mon>}}
\label{sec:org6967a6c}
列表生成式则可以用一行语句代替循环生成上面的list
\begin{verbatim}
print([x*x for x in range(1,11)])
print([x*x for x in range(1,11) if x%3==0])
print([m+n for m in 'abc' for n in 'xyz'])
import os
print(os.listdir('.'))
print([d for d in os.listdir('.')])
print('ABcd'.lower(), 'fdsa'.upper())
print('abc'[2])
L = ['Hello', 'World', 18, 'Apple', None]
print([s.lower() for s in L if isinstance(s, str)])
print([s.lower() if isinstance(s, str) else s for s in L])
\end{verbatim}

\subsubsection{generator 生成器 \textit{<2018-05-24 Thu>}}
\label{sec:org214c9ac}
受到内存限制,列表容量肯定是有限的,
在Python中,这种一边循环一边计算的机制,称为生成器:generator。
创建一个generator的方法:
\begin{itemize}
\item 第一种方法,只要把一个列表生成式的[]改成(),就创建了一个generator
\item 另一种,定义包含yield关键字的的函数, 在循环中不断yield.
如果想要拿到返回值,必须捕获StopIteration错误,
返回值包含在StopIteration的value中.
\end{itemize}
\#杨辉三角
\begin{verbatim}
#杨辉三角
def triangles():
    # n = 0
    # L1 = [1, 0]
    # while True:
    #     if n == 0 :
    #         yield L1[:-1]
    #     elif n == 1 :
    #         L1.insert(1, 1)
    #         yield L1[:-1]
    #     else:
    #         L = []
    #         L.insert(0, 1)
    #         for i in range(1, n):
    #             L.insert(i, L1[i-1] + L1[i])
    #         L.insert(len(L), 1)
    #         yield L
    #         L1 = L
    #     n = n + 1
    # return 'done'

    L = [1]
    yield L
    L = L + [1]
    yield L
    while True:
        L = [1] + [L[i] + L[i+1] for i in range(0, len(L) -1 )] + [1]
        yield L


n = 0
results = []
for t in triangles():
    print(t)
    results.append(t)
    n = n + 1
    if n == 10:
        break
if results == [
    [1],
    [1, 1],
    [1, 2, 1],
    [1, 3, 3, 1],
    [1, 4, 6, 4, 1],
    [1, 5, 10, 10, 5, 1],
    [1, 6, 15, 20, 15, 6, 1],
    [1, 7, 21, 35, 35, 21, 7, 1],
    [1, 8, 28, 56, 70, 56, 28, 8, 1],
    [1, 9, 36, 84, 126, 126, 84, 36, 9, 1]
]:
    print('测试通过!')
else:
    print('测试失败!')
\end{verbatim}

\begin{verbatim}
g=(x*x for x in range(9))
print(next(g))
print(next(g))
print(next(g))
print('????')
for i in g:
    print(i)
print()
def fibnacci(N):
    n,a,b=0,0,1
    while n < N:
        yield b
        a,b=b,a+b
        n=n+1
    return 'done'
f= fibnacci(9)
for i in f:
    print(i)
\end{verbatim}

\subsubsection{Iterator 迭代器 \textit{<2018-05-25 Fri>}}
\label{sec:org02ee4cb}
可以使用isinstance()判断一个对象是否是Iterable
可以被next()函数调用并不断返回下一个值的对象称为Iterator
生成器都是Iterator对象,但list、dict、str虽然是Iterable,却不是Iterator
把list、dict、str等Iterable变成Iterator可以使用iter()函数
Iterator的计算是惰性的,只有在需要返回下一个数据时它才会计算
\begin{itemize}
\item 凡是可作用于for循环的对象都是Iterable类型
\item 凡是可作用于next()函数的对象都是Iterator类型
\item Python的for循环本质上就是通过不断调用next()函数实现的
\end{itemize}
\begin{verbatim}
from collections import Iterable
print(isinstance([], Iterable))
print(isinstance({}, Iterable))
print(isinstance('abs', Iterable))
print(isinstance((x for x in range(9)), Iterable))
print(isinstance(range(9), Iterable))
from collections import Iterator
isinstance((x for x in range(10)), Iterator)

\end{verbatim}
\subsection{{\bfseries\sffamily TODO} 函数式编程}
\label{sec:org04e6682}
\begin{itemize}
\item State "TODO"       from ""           \textit{[2018-05-25 Fri 16:52]}
\end{itemize}
\subsection{module 模块\textit{<2018-05-25 Fri>}}
\label{sec:orge90aec7}
为了避免模块名冲突,Python引入了按目录来组织模块的方法,
称为Package(包)
\begin{itemize}
\item 一个abc.py的文件就是一个名字叫abc的模块,
\item 一个xyz.py的文件就是一个名字叫xyz的模块
\item 每一个包目录下面都会有一个\_\(_{\text{init}}\)\_\(_{\text{.py的文件}}\),这个文件是必须存在的
\item abc.py模块的名字就变成了mycompany.abc,类似的,
\item xyz.py的模块名变成了mycompany.xyz
\item \_\(_{\text{init}}\)\_\(_{\text{.py可以是空文件}}\),也可以有Python代码,
因为\_\(_{\text{init}}\)\_\(_{\text{.py本身就是一个模块}}\),而它的模块名就是mycompany
\item 可以有多级目录,组成多级层次的包结构
\item 模块名不要和系统模块名冲突,最好先查看系统是否已存在该模块,
检查方法是在Python交互环境执行import abc,若成功则说明系统存在此模块
\end{itemize}
\subsubsection{使用module \textit{<2018-06-06 Wed>}}
\label{sec:orgd647233}
\begin{verbatim}
#!/usr/bin/env python3
# -*- coding: utf-8 -*-

'this is a test module'

__author__ = 'ligy'


import sys

def test():
    args = sys.argv
    if len(args)==1:
        print('Hello, world!')
    elif len(args)==2:
        print('Hello, %s!' % args[1])
    else:
        print('Too many arguments!')

if __name__=='__main__':
    print('this is in main')
    test()
\end{verbatim}
\begin{itemize}
\item 任何模块代码的第一个字符串都被视为模块的文档注释
\item 使用\(_{\text{author}}\)\(_{\text{变量把作者写进去}}\)
\item argv至少有一个元素,因为第一个参数永远是该.py文件的名称
\item 这种if测试可以让一个模块通过命令行运行时执行一些额外的代码,
最常见的就是运行测试
\end{itemize}
\subsubsection{作用域 \textit{<2018-06-06 Wed>}}
\label{sec:org8cf4cf7}
\begin{itemize}
\item 有的函数和变量我们希望仅仅在模块内部使用
\item 类似\_\(_{\text{xxx}}\)\_\(_{\text{这样的变量是特殊变量}}\),可以被直接引用,但是有特殊用途,
比如上面的\_\(_{\text{author}}\)\_\_,\_\(_{\text{name}}\)\_\_
\item 类似\(_{\text{xxx和}}\)\_\(_{\text{xxx这样的函数或变量就是非公开的}}\)(private),
不应该被直接引用,比如\(_{\text{abc}}\),\_\(_{\text{abc等}}\)
\end{itemize}
\begin{verbatim}
def _private_1(name):
    return 'Hello, %s' % name

def _private_2(name):
    return 'Hi, %s' % name

def greeting(name):
    if len(name) > 3:
        return _private_1(name)
    else:
        return _private_2(name)
\end{verbatim}
\begin{itemize}
\item 调用greeting()函数不用关心内部的private函数细节,
这也是一种非常有用的代码封装和抽象的方法
\item 外部不需要引用的函数全部定义成private,
只有外部需要引用的函数才定义为public
\end{itemize}
\subsubsection{安装第三方模块 \textit{<2018-06-07 Thu>}}
\label{sec:org32b1075}
pip install pillow
\subsubsection{安装常用模块 \textit{<2018-06-07 Thu>}}
\label{sec:org6bc7635}
\begin{itemize}
\item anaconda (狂蟒之灾电影)
\item 默认情况下,Python解释器会搜索当前目录、所有已安装的内置模块和第三方模块,
搜索路径存放在sys模块的path变量中
\item 添加自己的搜索目录,有两种方法:
\begin{itemize}
\item sys.path.append('/Users/michael/my\(_{\text{py}}\)\(_{\text{scripts}}\)')
\item 设置环境变量PYTHONPATH
\end{itemize}
\end{itemize}
\subsection{面向对象编程 Object Oriented Programming}
\label{sec:orgf4ddb0b}
\begin{itemize}
\item OOP把对象作为程序的基本单元,一个对象包含了数据和操作数据的函数.
\item 在Python中,所有数据类型都可以视为对象,当然也可以自定义对象。
\item 自定义的对象数据类型就是面向对象中的类(Class)的概念。
\item 数据封装、继承和多态是面向对象的三大特点
\end{itemize}
\subsubsection{类与实例 Class Instance}
\label{sec:orgbcf8e71}
\begin{itemize}
\item 类名通常是大写开头的单词
\item 创建实例是通过类名+()实现的
\item My\(_{\text{students本身则是一个类}}\)
\item 可以自由地给一个实例变量绑定属性,比如,给实例bart绑定一个name属性
\item 有了\_\(_{\text{init}}\)\_\(_{\text{方法}}\),在创建实例的时候,就不能传入空的参数了,
必须传入与\_\(_{\text{init}}\)\_\(_{\text{方法匹配的参数}}\),但self不需要传
\item 除self之外,类的方法和普通函数没有什么区别,所以,你仍然可以用默认参数、
可变参数、关键字参数和命名关键字参数。
\item 这些封装数据的函数是和Student类本身是关联起来的,我们称之为类的方法
\item 义一个方法,除了第一个参数是self外,其他和普通函数一样。
\item 调用一个方法,只需要在实例变量上直接调用,除了self不用传递,其他参数正常传入
\end{itemize}

\begin{itemize}
\item 和静态语言不同,Python允许对实例变量绑定任何数据,也就是说,对于两个实例变量,
虽然它们都是同一个类的不同实例,但拥有的变量名称都可能不同
\end{itemize}

\begin{verbatim}
class My_students(object):
    pass
ligy = My_students()
ligy.name = 'Li Gaoyang'
ligy.age = 26
print(ligy.name, ligy.age)
jjkprint(My_students)
class My_students2():
    def __init__(self, name, score):
        self.name = name
        self.score = score
    def print_score(self):
        print('%s: %s' % (self.name, self.score))
mark = My_students2('Mark', '99')
john = My_students2('John Nash', 100)
john.age = 70
print(mark.name)
mark.print_score()
\end{verbatim}
\subsubsection{访问限制 \textit{<2018-06-29 Fri>}}
\label{sec:org272c06c}
\begin{itemize}
\item 外部代码还是可以自由地修改一个实例的name、score属性
\item bart.score = 99
\item 实例的变量名如果以\_\(_{\text{开头}}\),就变成了一个私有变量(private),
只有内部可以访问,外部不能访问
\item 在方法中,可以对参数做检查,避免传入无效的参数
\item 以双下划线开头,并且以双下划线结尾的,是特殊变量,
特殊变量是可以直接访问的,不是private变量
\item 一个下划线开头的实例变量名,但是,请视为私有变量,不要随意访问
\item 仍然可以通过\(_{\text{Student}}\)\_\(_{\text{name来访问}}\)\_\(_{\text{name变量}}\),但是强烈建议你不要这么干,
因为不同版本的Python解释器可能会把\_\(_{\text{name改成不同的变量名}}\)
\item 表面上看,外部代码“成功”地设置了\_\(_{\text{name变量}}\),但实际上这个\_\(_{\text{name变量和class内部的}}\)\_\(_{\text{name变量不是一个变量}}\)!
内部的\_\(_{\text{name变量已经被Python解释器自动改成了}}\)\(_{\text{Student}}\)\_\(_{\text{name}}\),而外部代码给bart新增了一个\_\(_{\text{name变量}}\)
\end{itemize}
\begin{verbatim}
class Student(object):
    def __init__(self, name, gender):
        self.name = name
        self.__gender = gender
    def set_gender(self, gender):
        self.__gender = gender
    def get_gender(self):
        return self.__gender
bart = Student('Bart', 'male')
if bart.get_gender() != 'male':
    print('测试失败!')
else:
    bart.set_gender('female')
    if bart.get_gender() != 'female':
        print('测试失败!')
    else:
        print('测试成功!')

\end{verbatim}

\subsubsection{继承与多态 \textit{<2018-07-05 Thu>}}
\label{sec:org5dbc53c}
\begin{itemize}
\item 继承最大的好处是子类获得了父类的全部功能
\item 当子类和父类都存在相同的方法时,子类的方法覆盖了父类的方法
\item 判断一个变量是否是某个类型可以用isinstance()判断
\item 如果一个实例的数据类型是某个子类,那它的数据类型也可以被看做是父类
\item 调用方只管调用,不管细节
\end{itemize}
“开闭”原:
\begin{itemize}
\item 对扩展开放:允许新增Animal子类
\item 对修改封闭:不需要修改依赖Animal类型的run\(_{\text{twice}}\)()等函数
\end{itemize}
静态语言 vs 动态语言
\begin{itemize}
\item 对于静态语言(例如Java)来说,如果需要传入Animal类型,
则传入的对象必须是Animal类型或者它的子类,否则,将无法调用run()方法
\item 对于Python这样的动态语言来说,则不一定需要传入Animal类型。
我们只需要保证传入的对象有一个run()方法就可以了
\item 动态语言的“鸭子类型”,它并不要求严格的继承体系,一个对象只要“看起来像鸭
,走起路来像鸭子”,那它就可以被看做是鸭子
\end{itemize}
小结
\begin{itemize}
\item 继承可以把父类的所有功能都直接拿过来,这样就不必重零做起,
子类只需要新增自己特有的方法,也可以把父类不适合的方法覆盖重写。
\item 动态语言的鸭子类型特点决定了继承不像静态语言那样是必须的。
\end{itemize}
\begin{verbatim}
class Animal(object):
    def run(self):
        print('Animal is running...')
class Dog(Animal):
    def run(self):
        print('Dog is running...')
dog1 = Dog()
dog1.run()
dog1.run()
\end{verbatim}

\subsubsection{获取对象信息 \textit{<2018-08-10 Fri>}}
\label{sec:org1f0ade8}
\begin{itemize}
\item 使用type()函数来判断对象类型
\item 可以使用types模块中定义的常量来判断一个对象是否是函数
\item 总是优先使用isinstance()判断类型,可以将指定类型及其子类"一网打尽"
\item 获得一个对象的所有属性和方法,可以使用dir()函数
\end{itemize}
\begin{verbatim}
print(type(123))
print(type(abs))
print(type('fda')==type(abs))
import types
def fn():
    pass
print(type(fn))
print(type(fn)==types.FunctionType)
print(type((x for x in range(9)))==types.GeneratorType)
print(type([x for x in range(9)]))
print(dir('dafdsa'))
\end{verbatim}

\section{Tensor flow 中文社区}
\label{sec:org1c314b4}
url:\url{http://www.tensorfly.cn/tfdoc/get\_started/introduction.html}
\begin{verbatim}
import tensorflow as tf
\end{verbatim}
\subsection{介绍}
\label{sec:org90bd633}
\subsubsection{intro\(_{\text{mnist}}\)}
\label{sec:orgbf281e2}
\begin{enumerate}
\item mnist 数据集 \textit{<2018-05-05 Sat>}
\label{sec:orgd12657b}
\begin{itemize}
\item mnist.train.images [6000, 28*28]
数据集的标签是0到9的数字,是一个one-hot vector, (只有一个元素为1,其他为0.)
0表示为 [1, 0, 0, 0, 0, 0, 0, 0, 0, 0]
\item mnist.train.labels [6000, 10]
\end{itemize}
\item softmax regression 介绍
\label{sec:orgb1cfec8}
\begin{itemize}
\item softmax模型可以用来给不同的对象分配概率。即使在之后,我们训练更加精细的模型时,
最后一步也需要用softmax来分配概率。
\item 为了用python实现高效的数值计算,我们通常会使用函数库,比如NumPy,
会把类似矩阵乘法这样的复杂运算使用其他外部语言实现。
不幸的是,从外部计算切换回Python的每一个操作,仍然是一个很大的开销。
\item Tensorflow不单独地运行单一的复杂计算,
而是让我们可以先用图描述一系列可交互的计算操作,然后全部一起在Python之外运行。
\end{itemize}
\end{enumerate}
\subsubsection{expert\(_{\text{mnist}}\)}
\label{sec:orgc870cb3}
\subsection{MNIST introduction}
\label{sec:orgf936e7b}
在机器学习模型设计时必须有一个单独的测试数据集不用于训练
而是用来评估这个模型的性能,从而更加容易把设计的模型推广
到其他数据集上(泛化)。
\section{Google机器学习速成课程}
\label{sec:org5cb3d6a}
url:\url{https://developers.google.cn/machine-learning/crash-course/}
\section{Python 命令笔记}
\label{sec:org3ba885d}
\subsection{\textit{<2018-05-21 Mon>}}
\label{sec:org275a2d7}
os.listdir('.')
'AbC'.lower()
'AbC'.upper()
'abc'[2]
for k, v in \{'x': 'A', 'y': 'B'\}.items():
    pass
L=[]
L.append(x)
L.pop()
L.pop(3)
d.pop('Jack')
\subsection{\textit{<2018-05-24 Thu>}}
\label{sec:org8c03e0c}
可迭代对象:
\begin{itemize}
\item list
\item tuple
\item range(9)
\item dictionary
\item set
\item str
\item generator
\begin{itemize}
\item 生成器
\item generator function
\end{itemize}
\end{itemize}
\section{Plot scientific data using python}
\label{sec:org889d86b}
\subsection{'Colors in the default prop\(_{\text{cycle}}\)}
\label{sec:org1f81da4}
\begin{verbatim}
import numpy as np
import matplotlib.pyplot as plt
prop_cycle = plt.rcParams['axes.prop_cycle']
colors = prop_cycle.by_key()['color']
lwbase = plt.rcParams['lines.linewidth']
thin = lwbase / 2
thick = lwbase * 3
fig, axs = plt.subplots(nrows=2, ncols=2, sharex=True, sharey=True)
for icol in range(2):
    if icol == 0:
        lwx, lwy = thin, lwbase
    else:
        lwx, lwy = lwbase, thick
    for irow in range(2):
        for i, color in enumerate(colors):
            axs[irow, icol].axhline(i, color=color, lw=lwx, label=color)
            axs[irow, icol].axvline(i, color=color, lw=lwy)
        plt.legend()
for irow in range(2):
    axs[irow, 0].yaxis.set_ticks(np.arange(0, 10, 2))
fig.suptitle('Colors in the default prop_cycle', fontsize='large')
#plt.show()
plt.savefig("Colors_in_the_default_prop_cycle.eps", dpi=None, facecolor='w', edgecolor='w', orientation='portrait', format="eps", bbox_inches='tight')
print(plt.rcParams['axes.prop_cycle'].by_key()['color'])
\end{verbatim}
\section{git@github.com:lijin-THU/notes-python.git}
\label{sec:org52c2968}
\subsection{基础}
\label{sec:orgada0f66}
\subsubsection{文件读写 \textit{<2018-09-04 Tue>}}
\label{sec:org04a0079}
url: \url{https://github.com/lijin-THU/notes-python/blob/master/02-python-essentials/02.21-file-IO.ipynb}
Control + RightClick
\subsection{numpy \textit{<2018-09-04 Tue 11:14>}}
\label{sec:orgbc9721d}
\subsubsection{overview \textit{<2018-09-08 Sat>}}
\label{sec:org091d3a4}
\begin{itemize}
\item numpy中array()方法与基本库中list有很大不同,注意不要混淆!!!
\item 修改array的形状,shape方法
\item *乘法仍然是对应元素的乘积,并不是按照矩阵乘法来计算
\item 矩阵乘法为A.dot(b)
\item numpy.linspace(start\(_{\text{point}}\), end\(_{\text{point}}\), point\(_{\text{number}}\))
\item 从数组中选择元素==
\end{itemize}
\begin{verbatim}
a = [1, 2, 3, 9]
import numpy
b = numpy.array(a)
print(a * 2)
print(b * 2)
#print(a.shape)
print(b.shape)
b.shape = 2,2
print(b)
print(b * b)
c = numpy.linspace(0, 2*numpy.pi, 21)
sin_c = numpy.sin(c)
print(c)
print(sin_c)
print(sin_c > 0.8)
print(c[sin_c > 0.8])
print('s')
print(c[[False, False, False, True, True, True, True, True, False, False, False, False
, False, False, False, False, False, False, False, False, False]])
print(sin_c[sin_c > 0.8])
\end{verbatim}
\subsubsection{matplotlib basics \textit{<2018-09-08 Sat 17:11>}}
\label{sec:org9575366}
\begin{itemize}
\item 清除、关闭图像 clf(), close(), close('all')
\end{itemize}
\begin{verbatim}
from numpy import *
from matplotlib.pyplot import *
t = linspace(0, 2*pi, 50)
x = sin(t)
y = cos(t)
figure()
plot(x)
figure()
plot(y)
\end{verbatim}

\subsubsection{numpy array \textit{<2018-09-08 Sat 18:12>}}
\label{sec:org088bd5a}
\begin{itemize}
\item 查看形状, np.shape(a)
\item 查看元素数目, np.size(a)
\item 查看数组维数, np.ndim(b)
\item 与列表不同,数组中要求所有元素的 dtype 是一样的,
如果传入参数的类型与数组类型不一样,需要按照已有的类型进行转换
\item 多维数组索引, a[1, 3], a[1]
\item 切片是引用, 因此,改变 b 会改变 a 的值。与列表(list)不同
\item np.arange 函数来产生等差数组
\item 花式索引需要指定索引位置, a[\[1, 2, -3\]]。还可以使用布尔数组来花式索引。
但是list不行。
\item 与切片不同,花式索引返回的是原对象的一个复制而不是引用
\item where 函数会返回所有非零元素的索引
\end{itemize}
\begin{verbatim}
import numpy as np
a = [i for i in range(10)]
a.append('dog')
aa = [i for i in range(10)]
print(a)
b = np.array(a)
bb = np.array(aa)
print(b.dtype, b.itemsize, b.shape, np.shape(a), np.ndim(a))
c = np.arange(0, 10, 2)
d = range(0, 10, 2)
print(type(c), c > 3, b[[0, 3, 2]], np.where(c[c > 3]))
print(type(d))
\end{verbatim}
\subsubsection{array types \textit{<2018-09-09 Sun 15:47>}}
\label{sec:org0f4876b}
\begin{itemize}
\item 复数数组
\item 复共轭 a.conj()
\end{itemize}
\begin{verbatim}
import numpy as np
a = np.array([1 + 1j, 2, 3, 4])
print(a.real, a.imag, a.conj())
\end{verbatim}
\subsubsection{array calculation method \textit{<2018-09-09 Sun 16:57>}}
\label{sec:org6da5cb8}
\begin{itemize}
\item 求所有元素的和 sum(a)
\item 沿着第一维求和 sum(a, axis=0)
\item 沿着第二维求和 sum(a, axis=1)
\item 求所有元素的乘积 prod(a), a.prod()
\item 全局最小 a.min()
\item 沿着某个轴的最小 a.min(axis=1)
\item 最大最小值的位置 a.argmin(), a.argmax()
\item 均值 a.mean(axis=1)
\item 加权平均 a.average(axis=1), a.average(axis=0, weights=[1,2])
\end{itemize}
\subsubsection{numpy sorting \textit{<2018-09-09 Sun>}}
\label{sec:org236b0f3}
\subsection{{\bfseries\sffamily TODO} interfacing with other languages \textit{<2018-10-03 Wed>}}
\label{sec:org3a109b1}
\begin{itemize}
\item State "TODO"       from              \textit{[2018-10-03 Wed 20:15]}
\end{itemize}
\section{Coursera-ML-AndrewNg-Notes}
\label{sec:orgd48c406}
\subsection{\textit{<2019-03-30 Sat 01:00> } 课时12 梯度下降}
\label{sec:orgecaa62c}
\begin{itemize}
\item alpha是learning rate,可以是常数,不能太大或太小。
\item 应该同时更新Theta\(_{\text{0,Theta}}\)\(_{\text{1,\ldots{}Theta}}\)\(_{\text{n}}\)
\item 事实证明线性回规的cost function都是弓形函数(凸函数,convex function),
只有一个全局最小,没有局部最小。
\item batch Gradient Descent: 每次更新时遍历所有training set.
\item 也可以不遍历所有training set.
\item Normal equation方法不适用于大量training set,梯度下降可以。
\end{itemize}
\end{document}